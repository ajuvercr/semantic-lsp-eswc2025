\documentclass[runningheads]{llncs}
\usepackage[T1]{fontenc}
\usepackage{algpseudocode, algorithm, amsmath, amssymb}					% packages to get the fonts, symbols used in most math
%\usepackage{dcolumn}
\usepackage[english]{babel}
\usepackage{hyperref}
\usepackage{multirow}
\usepackage{latexsym}
\usepackage{listings}
\usepackage{booktabs}
\usepackage{tabularx} % For dynamic table width
\newcolumntype{C}{>{\centering\arraybackslash}X}
\renewcommand{\arraystretch}{1.5}
\usepackage{pifont} % For check and cross symbols
\newcommand{\cmark}{\textcolor{darkgreen}{\ding{51}}}  % Checkmark
\newcommand{\xmark}{\textcolor{red}{\ding{55}}}  % Cross
\newcommand{\mmark}{\textcolor{orange}{\ding{106}}}  % Cross

% T1 fonts will be used to generate the final print and online PDFs,
% so please use T1 fonts in your manuscript whenever possible.
% Other font encondings may result in incorrect characters.
%
\usepackage{graphicx}
% Used for displaying a sample figure. If possible, figure files should
% be included in EPS format.
%
\usepackage[dvipsnames,svgnames, table]{xcolor} % text color
\DeclareFontFamily{U}{mathx}{\hyphenchar\font45}
\DeclareFontShape{U}{mathx}{m}{n}{
      <5> <6> <7> <8> <9> <10>
      <10.95> <12> <14.4> <17.28> <20.74> <24.88>
      mathx10
      }{}
\DeclareSymbolFont{mathx}{U}{mathx}{m}{n}
\DeclareMathSymbol{\bigtimes}{1}{mathx}{"91}



\DeclareRobustCommand{\ojoin}{\rule[0.10ex]{.3em}{.4pt}\llap{\rule[1.40ex]{.3em}{.4pt}}}
\newcommand{\leftouterjoin}{\mathrel{\ojoin\mkern-6.5mu\Join}}
\newcommand{\rightouterjoin}{\mathrel{\Join\mkern-6.5mu\ojoin}}
\newcommand{\fullouterjoin}{\mathrel{\ojoin\mkern-6.5mu\Join\mkern-6.5mu\ojoin}}


\definecolor{eclipseStrings}{RGB}{42,0.0,255}
\definecolor{eclipseKeywords}{RGB}{127,0,85}
\definecolor{darkgreen}{RGB}{0,128,21}
\colorlet{numb}{magenta!60!black}
\setlength{\parskip}{0pt}

\algrenewcommand\algorithmicrequire{\textbf{Input:}}
\algrenewcommand\algorithmicensure{\textbf{Returns:}}


\lstdefinelanguage{sparql}{
    basicstyle=\tiny,
    keywordstyle=\color{eclipseKeywords},
    commentstyle=\color{blue}, % style of comment
    stringstyle=\color{darkgreen},
    keywords={FILTER, NOT, EXISTS, bound, OPTIONAL, BIND, IF, isIRI, AS, INSERT, UPDATE, DELETE, WHERE, SELECT, UNION, rr,rml,ql,rdfs},
    morecomment=[n]{?}{\ },
    morecomment=[l]\#,
    morestring=[b]",
    morestring=[s]{<}{>},
    frame=lines,
}
\lstdefinelanguage{rdf}{
    basicstyle=\tiny,
    keywordstyle=\color{eclipseKeywords},
    commentstyle=\color{gray}, % style of comment
    stringstyle=\color{darkgreen},
    keywords={rml, ql, rr, rdfs},
    morecomment=[n]{?}{\ },
    morecomment=[l]\#,
    morestring=[b]",
    morestring=[s]{<}{>},
    frame=lines,
}


\lstdefinelanguage{triples}{
    basicstyle=\tiny,
    keywordstyle=\color{eclipseKeywords},
    commentstyle=\color{darkgreen}, % style of comment
    keywords={a, rml, ql, rr, rdfs},
    morecomment=[s]{?}{\ },
    frame=lines,
}

  

% Set symbols % 
\newcommand{\symFragmentUniverse}{\mathcal{F}}
\newcommand{\symAttrUniverse}{\mathcal{A}}
\newcommand{\symTermUniverse}{\mathcal{T}}
\newcommand{\symAllStrings}{\mathcal{S}} % the symbol for the set of all strings
\newcommand{\symAllIRIs}{\mathcal{I}} % the symbol for the set of all IRIs
\newcommand{\symAllLiterals}{\mathcal{L}} % the symbol for the set of all literals
\newcommand{\symAllBNodes}{\mathcal{B}} % the symbol for the set of all blank nodes
\newcommand{\symRMLGraph}{G}
\newcommand{\symDataObjUni}{\mathcal{D}}
\newcommand{\symDataAccUni}{\mathcal{Q}}
\newcommand{\symDataSeqUni}{\bar{\mathcal{O}}}
\newcommand{\symAttrSubSet}{A}
\newcommand{\symProjSet}{P}
\newcommand{\symMappingTupleSet}{M}
\newcommand{\symRenamePFunc}{R}
\newcommand{\bigDataObject}{D}
\newcommand{\symSetDataset}{\mathcal{D}^\texttt{\tiny ds}\!}
\newcommand{\symSetDatasetX}[1]{\mathcal{D}^\texttt{\tiny ds}_{#1}}
\newcommand{\symSetContentA}{\mathcal{D}^\texttt{\tiny c1}}
\newcommand{\symSetContentAX}[1]{\mathcal{D}^\texttt{\tiny c1}_{#1}}
\newcommand{\symSetContentB}{\mathcal{D}^\texttt{\tiny c2}}
\newcommand{\symSetContentBX}[1]{\mathcal{D}^\texttt{\tiny c2}_{#1}}
\newcommand{\symDataAcc}{L}
\newcommand{\symDataAccX}[1]{L_{#1}}
\newcommand{\symDataSeqCard}[1]{D^{seq,#1}}
\newcommand{\symAttrQueryMap}{\mathbb{P}}
\newcommand{\symDataSeq}{\bar{O}}
\newcommand{\symExtExprSet}{E}
\newcommand{\symExtExprSeq}{\bar{\symExtExprSet}}
\newcommand{\symExtPairs}{\symExtExprSet}
\newcommand{\symTermSeqUniverse}{\bar{\symTermUniverse}}
\newcommand{\symFragmentPFunc}{\bbbf}
\newcommand{\symFragmentSet}{F}
\newcommand{\symJoinAttrPairs}{\mathbb{J}}
\newcommand{\symExtFuncUni}{\mathcal{E}}

% Special values % 
\newcommand{\card}{card}
\newcommand{\fragAttr}{a_\mathrm{frag}}
\newcommand{\fragDefault}{f_{0}}
\newcommand{\attr}{a}
\newcommand{\subjAttr}{\attr_\textrm{s}}
\newcommand{\predAttr}{\attr_\textrm{p}}
\newcommand{\objAttr}{\attr_\textrm{o}}
\newcommand{\graphAttr}{\attr_\textrm{g}}
\newcommand{\fragment}[0]{fragment}
\newcommand{\error}{\epsilon}
\newcommand{\mappingTuple}{t}
\newcommand{\symMappingInst}{I}
\newcommand{\mappingRel}{(\symAttrSubSet, \symMappingInst)}
\newcommand{\mappingRelSpec}[1]{(\symAttrSubSet_{#1}, \symMappingInst_{#1})}
\newcommand{\queryExpr}{q}
\newcommand{\rootIt}{rit}
\newcommand{\cobit}{cobit}
\newcommand{\dataObject}{d}
\newcommand{\eval}{\mathit{eval}}
\newcommand{\evalX}[1]{\eval_{#1}}
\newcommand{\cbeval}{\mathit{eval}'\!}
\newcommand{\cbevalX}[1]{\eval_{#1}'}
\newcommand{\sourceType}{type}
\newcommand{\sourceTypeX}[1]{type_{#1}}
\newcommand{\sourceTypeTuple}{(\symSetDataset, \symSetContentA\!, \symSetContentB\!, \symDataAcc, \symDataAcc'\!, \eval, \cbeval, \typeCast)}
\newcommand{\sourceTypeTupleX}[1]{\sourceTypeTupleXXX{#1}{#1}{#1}}
\newcommand{\sourceTypeTupleXXX}[3]{(\symSetDatasetX{#1}, \symSetContentAX{#2}, \symSetContentBX{#3}, \symDataAccX{#2}, \symDataAccX{#3}', \evalX{#2}, \cbevalX{#3}, \typeCastX{#1})}
\newcommand{\dataSource}{s}
\newcommand{\dataSourceTuple}{(\sourceType,\bigDataObject)}
\newcommand{\iterConfig}{\bbbc}
\newcommand{\sourceOp}[3]{\textrm{Source}^{(#1,#2,#3)}}
\newcommand{\sourceOpDflt}{\sourceOp{\dataSource}{\queryExpr}{\symAttrQueryMap}}
\newcommand{\sourceOpTuple}{(\dataSource, \iterConfig)}
\newcommand{\attrQueryPair}{(\attr \rightarrow \queryExpr)}
\newcommand{\typeCast}{cast}
\newcommand{\typeCastX}[1]{\typeCast_{#1}}
\newcommand{\length}{len}
\newcommand{\var}{v}
\newcommand{\indexSeq}{M}
\newcommand{\extExpr}{\varphi}
\newcommand{\extEval}{eval}
\newcommand{\extPairsType}{(\symAttrUniverse \setminus \symAttrSubSet) \times \symExtExprSeqUni}
\newcommand{\serdeAttr}{a_{serialized}}
\newcommand{\bgp}{\bbbb}
\newcommand{\extFunc}{f}
\newcommand{\extFuncType}{(\symTermUniverse \cup \{\error\})}
\newcommand{\extFuncTuple}{(\extFunc,\extExpr_1,\dots,\extExpr_n)}
\newcommand{\extOp}{\text{Extend}_{\varphi}^{\attr}}
\newcommand{\extOpAttr}[1]{\text{Extend}_{\varphi}^{#1}}
\newcommand{\renameOp}{\text{Rename}^{\symRenamePFunc}}
\newcommand{\equiJoinOp}{\text{Join}^{\symJoinAttrPairs}_{=}}
\newcommand{\triplesMapIRI}{u}
\newcommand{\triplesMapGraph}[2]{\symRMLGraph_{#1}^{#2}}
\newcommand{\toIRI}{\texttt{toIRI}}
\newcommand{\toBNode}{\texttt{toBNode}^{\LTB}}
\newcommand{\toLiteral}{\texttt{toLiteral}}
\newcommand{\template}{\texttt{template}^{\symAttrQueryMap}}
\newcommand{\initSrc}{\texttt{initSrc}}
\newcommand{\concat}{\texttt{concat}}
\newcommand{\concatSeq}{\texttt{concatSeq}}
\newcommand{\literal}{\ell}
\newcommand{\lex}{\mathit{lex}}
\newcommand{\dt}{\mathit{dt}}
\newcommand{\literalTuple}{(\lex, \dt)}
\newcommand{\LTB}{\mathit{L2B}}
\newcommand{\tempSubStrs}{\texttt{tempSubStrs}}
\newcommand{\substrSeq}{\bar{S}}
\newcommand{\projectOp}[1]{\text{Project}^{#1}}
\newcommand{\natJoin}{\text{NatJoin}}
\newcommand{\union}{\text{Union}}

% Functions % 
\newcommand{\fctDom}[1]{\mathrm{dom}(#1)}
\newcommand{\fctMappingTuple}[1]{\mappingTuple(#1)}
\newcommand{\fctAttrs}[1]{\mathrm{attrs}(#1)}
\newcommand{\fctCard}[2]{\card[#1](#2)}
\newcommand{\fctRootIt}[2]{\rootIt(#1, #2)}
\newcommand{\fctCobit}[3]{\cobit(#1, #2, #3)}
\newcommand{\fctEval}[2]{\eval(#1,#2)}
\newcommand{\fctEvalLang}[3]{\eval^{#3}(#1,#2)}
\newcommand{\fctCbeval}[3]{\cbeval(#1,#2,#3)}
\newcommand{\fctCbevalLang}[4]{\cbeval^{#4}(#1,#2,#3)}
\newcommand{\fctTypeCast}[1]{\typeCast(#1)}
\newcommand{\fctLength}[1]{\length(#1)}
\newcommand{\fctRenameTuple}[2]{rename^{#1}(#2)}
\newcommand{\fctSubst}[2]{subst^{#1}(#2)}
\newcommand{\fctExtFuncEvalN}[2]{\extFunc(\fctEval{#1_1}{#2},\dots,\fctEval{#1_n}{#2})}
\newcommand{\fctExtOp}[1]{\extOp(#1)}
\newcommand{\fctExtOpBig}[1]{\extOp\bigl(#1\bigr)}
\newcommand{\fctExtOpX}[3]{\text{Extend}^{#1}_{#2}(#3)}
\newcommand{\fctExtOpXBig}[3]{\text{Extend}^{#1}_{#2}\bigl(#3\bigr)}
\newcommand{\fctRenameOp}[1]{\renameOp(#1)}
\newcommand{\fctRenameOpBig}[1]{\renameOp\bigl(#1\bigr)}
\newcommand{\fctEquiJoinOp}[2]{\equiJoinOp(#1, #2)}
\newcommand{\fctEquiJoinOpBig}[2]{\equiJoinOp\bigl(#1, #2\bigr)}
\newcommand{\fctToIRI}[1]{\toIRI(#1)}
\newcommand{\fctToBNode}[1]{\toBNode(#1)}
\newcommand{\fctToLiteral}[1]{\toLiteral(#1)}
\newcommand{\fctTemplate}[1]{\template(#1)}
\newcommand{\fctImg}[1]{\mathrm{img}(#1)}
\newcommand{\fctInitSrc}[1]{\initSrc(#1)}
\newcommand{\fctConcat}[2]{\concat(#1, #2)}
\newcommand{\fctConcatSeq}[1]{\concatSeq(#1)}
\newcommand{\fctTempSubStrs}[1]{\tempSubStrs(#1)}
\newcommand{\fctProjectOpDflt}[1]{\fctProjectOp{\symProjSet}{#1}}
\newcommand{\fctProjectOp}[2]{\projectOp{#1}(#2)}
\newcommand{\fctProjectOpBig}[2]{\projectOp{#1}\bigl(#2\bigr)}
\newcommand{\fctNatJoin}[2]{\natJoin(#1,#2)}
\newcommand{\fctUnion}[2]{\union(#1,#2)}

% Layout %
\newcommand{\ttl}[1]{\texttt{\small #1}}  % for conrete IRIs, etc. used in examples and in formulas



%%%%%%%%%%%%%%%%%%%%%%%%%%%%%%%%%%%%%%%%%%%%%%%
% The following ensures that we have a (non-visible) table of contents embedded
% in the PDF, which PDF readers can show and, thus, allows me to navigate the
% document more easily.
%                                    Olaf
\setcounter{tocdepth}{2}
\hypersetup{bookmarksopen=true, citecolor=blue}
%
%%%%%%%%%%%%%%%%%%%%%%%%%%%%%%%%%%%%%%%%%%%%%%%


\begin{document}												% end of preamble and beginning of text that will be printed

\title{Semantic Languages servers are cool}
%
%\titlerunning{Abbreviated paper title}
% If the paper title is too long for the running head, you can set
% an abbreviated paper title here
%
\author{Arthur Vercruysse\inst{1}\orcidID{0000-0000-0000-0000} \and
Julian\inst{1}\orcidID{0000-0002-1741-2090} \and 
Pieter Colpaert\inst{1}\orcidID{0000-0002-1741-2090}}
%

\authorrunning{A. Vercruysse et al.}
% First names are abbreviated in the running head.
% If there are more than two authors, 'et al.' is used.
%
\institute{
University of Ghent - imec, Ghent, Belgium \\
\email{arthur.vercruysse@ugent.be}}
%
\maketitle              % typeset the header of the contribution
%
% \begin{abstract}
% % Context (What is needed to understand the "need"?)
%   The semantic web has produced many syntaxes to interact with it: from data format to querying to reasoning.
% % Need (Why something needed to be done at all?)
%   These formats all suffer from human imprecisions, a single typo changes the entire semantics of the document, leaving it non-interoperable.
% % Task (What was undertaken to address the need? It’s here that you write ‘in this paper, we …’)
%   In this paper, we introduce the semantic web language server.
% % Object (What the present document does or covers)
%   The language server enhances semantic documents with IDE functionality. 
%   Notifying the users early about potential mistakes from typo's to SHACL violations.
% % Findings (What the work done yielded or revealed)
%   By combining and extending the state of the art, we enhanced the efficienty, precision and confidance of end users working with semantic documents,
%   including power users, newcomers, domain experts and data engineers.
% % Conclusion (What the findings mean for the audience)
%   With the semantic web language server users can expect the similar functionality as existing tools like Yasgui, but closer the user in their coding environment.
% % Perspectives (What the future holds, beyond this work)
% \keywords{Language Server, IDE, Tool}
% \end{abstract}


\begin{abstract}
% Context (What is needed to understand the "need"?)
The Semantic Web has introduced a variety of syntaxes for e.g., serializing, querying, and validating linked data, such as Turtle, SPARQL, and SHACL.
% Need (Why something needed to be done at all?)
While these formats enable powerful interactions with data, they are highly sensitive to human error; even minor typos can disrupt the semantics of a document, rendering it invalid or non-interoperable.
% Task (What was undertaken to address the need? It’s here that you write ‘in this paper, we …’)
In this paper, we study how the authoring experience of Semantic Web documents can be enhanced through the use of the Language Server Protocol (LSP) with for instance code completion, syntax highlighting and live validation output.
% Object (What the present document does or covers)
To that extent, we introduce the Semantic Web Language Server (SWLS), an LSP implementation with features such as real-time syntax validation, context-aware autocompletion, and SHACL-based diagnostics to notify users of potential mistakes when interacting with Semantic Web documents.
% The language server is available as extensions for popular platforms like VS Code and Neovim, as well as in a standalone web-based interface (integrated into a Monaco editor).
% Findings (What the work done yielded or revealed)
By extending functionalities beyond what is already supported by the best-in-class YASGUI interface, our tool aims to further improve the development efficiency, precision, and confidence of power users, newcomers, domain experts, and data engineers.
% Conclusion (What the findings mean for the audience)
It integrates seamlessly into established Web-based and desktop development environments, and its layered architecture facilitates extending the code base to support new features in the future. 
% PC: This is a main track resource paper, not a workshop paper. Don’t promise an evaluation in future work.
% In future work, we will further extend feature support and perform user evaluations that allow us to identify improvement points. 
% Perspectives (What the future holds, beyond this work)
% This work not only addresses current limitations in semantic web tooling but also paves the way for broader adoption of semantic technologies by reducing barriers and improving usability.
\keywords{Language Server, IDE, Tool, End User Software Engineering, Semantic Web}
\end{abstract}

\textbf{Resource Types:} An open-source implementation and a Web-based demo

\textbf{URL:} \url{https://github.com/ajuvercr/semantic-web-lsp} 

\textbf{Licenses:} MIT License

% \textbf{DOI:} [TODO add DOI] 

\section{Introduction}%
\label{sec:introduction}

\begin{itemize}
\item Context: sometimes you have to edit RDF files: turtle, jsonld, or query it using hand written sparql queries.
\item Problem: humans make mistakes, which may result in big problems (making a mistake in a prefix statement changes all related triples)
\item Problem: difficult to learn new properties/ontologies. No autocompletion makes a user guessing
\item Problem: iterating on documents is slow: 
  \begin{itemize}
    \item shacl violations need an extra step during iteration
    \item inconsistent types
  \end{itemize}
\item related work: many iterations are already implemented for specialized use cases.
  \begin{itemize}
    \item \href{https://query.wikidata.org/}{wikidata}
    \item \href{query.linkeddatafragments.org}{linkeddatafragments}
    \item \href{https://docs.triply.cc/yasgui/}{yasgui} \cite{yasgui}
  \end{itemize}
\item Solution: semantic LSP
\end{itemize}

% \section{Related Work}%
% \label{sec:related_work}
%
% In this section we look at the related state of the art.
% The state of the art is full of different implementations that handle some specific part of the semantic web.
% From ontology designers like protege to Yasgui's sparql query editors.
% This section covers the different IDE functionalities that can be implemented for the semantic web, Table 1 provides details on current and open-source implementations that implement parts of these functionalities.
%
% \paragraph*{Highlighting}
%
% Highlighting helps human brains parse data more easily.
% There are two different parts of highlighting: syntactic highlighting and semantic highlighting.
% The former only concerns itself with the syntax of a document. A lexer can handle this highlighting very effectively and allows for example to hightlight strings as green and namednodes as red.
% Semnatic highlighting on the other hand can use document semantics. Semantics can be for example the term kind of a term, only inferable by extracting the actual triples or marking the type of an object in a different colour.
%
% Please note that undefined properties etc should not be highlighed differently by the semantic highlighter, undefined properties should be handled as diagnostics.
%
%
% \paragraph*{Diagnostics}
%
% Diagnostics are of utmost importance, nothing is more draining than to check whether or not some data is correctly written at runtime.
% The editor should be able to notify the user that the current document would not be parsed by parsers following the standard.
%
% Other diagnostics are also very relevant and can inform the user that the current data may not be what they expect.
% For example the editor can notify the user when an undefined property is used, alerting the user that it could be a typo.
% The editor can also notify the user about reasoning mistakes, when creating an ontology or creating data that should adhere to an ontology or to a SHACL shape.
%
%
% \paragraph*{Formatting}
%
% It is expected that an IDE helps the developer to keep a consistent format over different documents, again helping the user parse the data more effectively.
% Formatting however is a subjective matter and should be configurable by the user or codebase.
%
% \paragraph*{Renaming}
% Often an editor allows the user to rename identifiers as a quality of life improvemt.
% This allows the user to more easily iterate over data and aliviating the writer's block.
%
%
% \paragraph*{Completion}
%
% Completion comes again in two flavours: completing keywords and already defined tokens and semantic completion.
% Semantic completion can complete with defined properties and defined classes and can check the current context of the completion, answering the question "Is the user current writing a class or a property?".
%
% Naive completion can suggest all defined properties and defined classes but this is not always wanted, the editor should also be able to derive the type of the current subject for example and only suggest properties that are defined with that type as domain.
% On the other hand when the user is writing an object, the editor can also suggest already defined entities that have the correct type checking the range of the property.
%
%
%
%
%
% \subsection*{Some words on ontology designing tools}
% Their main helping functions~\cite{ComparingOntologyBuildingTools}.
%
% \subsection{Language Server Protocol}
%
% Language Servers using the Language Server Protocol~\cite{IntroToLsp} 
% are cool ~\cite{GLSPFlexibility}.
%


%% GPT

\section{State of the Art}%
\label{sec:related_work}

%PC: I’d consider renaming related work state of the art then!
This section examines the state of the art in tools and functionalities that support developement and interaction with Semantic Web technologies. 
Current implementations address various aspects of Semantic Web technologies, including for example, ontology modeling tools (e.g., Protégé) and SPARQL query editors (e.g., YASGUI).
Next we describe a set of common IDE functionalities~\cite{HowAreJava} and discuss existing tools that (partially) cover them for Semantic Web-related formats.

\paragraph*{Highlighting} enhances readability by leveraging visual cues, which assist developers in understanding complex data.
For example by colouring differently reserved keywords (e.g., \texttt{a}, \texttt{@prefix} in the RDF Turtle syntax) and data types (e.g., strings, numbers, etc.).  
% Semantic highlighters should avoid overloading the user with unnecessary information. 
% Undefined properties, for example, should not be highlighted differently; such cases are better addressed by diagnostic features.~\todoCite{This needs to be backed by a citation. Otherwise just limit to explain what is highlighting and why is important.}
There are two primary types of highlighting: \textbf{syntactic} and \textbf{semantic}.

\begin{itemize}
    \item \textbf{Syntactic highlighting} focuses on document syntax and is implemented using lexers (i.e. software that transforms text into tokens: keyswords, IRIs, blanknode identifiers, etc.).
      For example in RDF Turtle, strings might appear in green, while named nodes are highlighted in red.
      This type of highlighting does not require knowledge of the document's meaning.
      Syntactic highlighting is enabled on all software we considered, except for the 
      \textit{rdf-vocabularies-autocomplete} VSCode extension, VSCode handles syntactic highlighting already with the Tree-sitter project\footnote{\url{https://tree-sitter.github.io/tree-sitter/}}.
    \item \textbf{Semantic highlighting}, in contrast, uses the local (and externally linked) semantics referenced in the document to differentiate terms based on their roles. 
      For instance, terms declared in an ontology can be highlighted differently compared to locally defined terms.
      We found no semantic web tooling that implemented this feature.
\end{itemize}


\paragraph*{Diagnostics} are essential for identifying issues early in the development process. 
Rather than deferring error detection to runtime, an IDE should provide immediate feedback on data validity and correctness.
These features enable developers to maintain data accuracy and reduce time spent debugging. 
The stardog language servers and the Yasgui query editor notify the users of syntax errors.

% Examples of valuable diagnostics include:
\begin{itemize}
    \item \textbf{Standard compliance}: Notifying users if the current document is incompatible with parsers adhering to Semantic Web standards.
    \item \textbf{Undefined properties}: Alerting users to potential typing mistakes or incorrect terms when undefined properties are detected.
    \item \textbf{Reasoning errors}: Highlighting logical inconsistencies in ontologies or data that deviate from expected SHACL shapes or OWL reasoning constraints.
\end{itemize}


\paragraph*{Formatting} 
Consistent formatting across documents can help enhancing readability and maintainability.
IDEs assist users in adhering to a defined style guide while allowing customization to accommodate different workflows or preferences.
We found no semantic web tooling that implemented this feature.

\paragraph*{Renaming} identifiers, such as classes or properties, is an important feature for iterative development.
This functionality helps users manage data more efficiently and adapt to evolving project requirements without introducing errors.
We found no semantic web tooling except for our previous work~\cite{JSONLD-LSP}, that handles renaming identifiers.

\paragraph*{Completion} features, similar to highlighting, can be categorized as \textbf{syntactic} or \textbf{semantic}:
Simple semantic completion might suggest all possible terms, but a sophisticated system should filter suggestions based on the context, such as the inferred type of the current subject or object.
All semantic web tooling we considered, included some kind of completion mechanism, none however applied typed completion.


\begin{itemize}
    \item \textbf{Syntactic completion} involves suggesting keywords or tokens already defined in the document.
    \item \textbf{Semantic completion} provides context-aware suggestions, such as properties or classes relevant to the current position. For example:
    \begin{itemize}
        \item When writing a property in an RDF document, the IDE can suggest those that define the subject's type as their \texttt{rdfs:domain}.
        \item When completing an object, it can suggest entities consistent with the property's \texttt{rdfs:range}.
    \end{itemize}
\end{itemize}

In Table~\ref{tab:current_implementations} we show a summary of exisiting open-source tools that implement at least one of the aforementioned IDE functionalities for Semantic Web technologies.
% PC: This is something you’d like the reader to accept indeed when you introduced the SOTA entirely. I’d put this sentence at the end.
While these tools provide specialized capabilities, the integration of features typical of IDEs remains limited.
With SWLS we aim on addressing this gap and in general, support the broader adoption of Semantic Web technologies. 

% we might want to transpose the headers
\begin{table}[h!]
    \centering
  \begin{tabularx}{\textwidth}{ |p{3cm}|C|C|C|C|C|C|C|C|C|C|C|C|}
\hline
    \multirow{2}{*}{} & \multicolumn{2}{c|}{Highlight} & \multicolumn{3}{c|}{Diagnostics} & \multicolumn{4}{c|}{Completion}  & \multirow{2}{*}{} & \multirow{2}{*}{} & \multirow{2}{*}{} \\ \cline{2-10} 

      LSP implementation                & \rotatebox{90}{Syntactic} & \rotatebox{90}{Semantic} %highlighting
                                        & \rotatebox{90}{Syntax} & \rotatebox{90}{Undefined} & \rotatebox{90}{Validation\ \ }  %diagnostics
                                        & \rotatebox{90}{Syntax} & \rotatebox{90}{Tokens} & \rotatebox{90}{Simple} & \rotatebox{90}{Typed} % completion
                                        & \rotatebox{90}{Formatting} 
                                        & \rotatebox{90}{Renaming}
                                        & \rotatebox{90}{Polyglot} \\ \hline
Stardog                       & \cmark & \xmark & \cmark & \xmark & \xmark & \cmark & \cmark & \mmark & \xmark & \xmark & \xmark & \cmark \\
RDFox                         & \cmark & \xmark & \xmark & \xmark & \xmark & \mmark & \xmark & \xmark & \xmark & \xmark & \xmark & \cmark \\
query.wikidata                & \cmark & \xmark & \xmark & \xmark & \xmark & \xmark & \cmark & \cmark & \xmark & \xmark & \xmark & \xmark \\
yasgui                        & \cmark & \xmark & \cmark & \xmark & \xmark & \xmark & \cmark & \cmark & \xmark & \xmark & \xmark & \xmark \\
Protégé                       & N/A    & N/A    & N/A    & N/A    & \cmark & N/A    & \cmark & N/A    & N/A    & N/A    & \cmark & \xmark \\
rdf-vocabularies-autocomplete & N/A    & N/A    & N/A    & N/A    & N/A    & \xmark & \xmark & \cmark & \xmark & N/A    & N/A    & \xmark \\
JSON-LD LSP                   & \cmark & \xmark & \cmark & \xmark & \xmark & \xmark & \xmark & \cmark & \xmark & \xmark & \cmark & \xmark \\
SWLS                          & \cmark & \cmark & \cmark & \cmark & \cmark & \cmark & \cmark & \cmark & \cmark & \cmark & \cmark & \cmark \\
\hline
\end{tabularx}
    \caption{\label{tab:current_implementations}
    \mmark Stardog simple completion is based on a fixed list of items. 
    \mmark RDFox syntax completion is only based on predefined for SPARQL functions.
    Table listing IDE features of different Semantic Web tools.
    All eligable software supports both highlighting and completion in one form of another, 
    while no software supports formatting or renaming except for Protégé and JSONLD-LSP (our previous work). 
  }
\end{table}


\subsection*{Ontology Design Tools}

Ontology design tools, such as Protégé~\cite{protege2015}, serve as the foundation for many Semantic Web development workflows. 
Their primary functions include visualizing ontologies, defining classes and properties, and validating logical consistency through reaoning.
These tools provide essential support but often lack the integrated functionality of modern IDEs, such as context-aware diagnostics or advanced refactoring capabilities~\cite{ComparingOntologyBuildingTools}.



\section{Semantic LSP}%
\label{sec:semantic_lsp}

When developing a LSP, we made sure to follow the state of the art best implementation practices\cite{10.1145/3550355.3552452,10.1145/3563834.3567537,10.1145/3550355.3552452,Bour_2018}.
Semantic LSP is built on top of an entity component system, allowing for even greater seperation of concern than the proposed \textit{Layered Architecture}\cite{10.1145/3550355.3552452}.

Entity component system is a software pattern that involves breaking your program up into Entities, Components, and Systems.
Entities are unique "things", here documents, that are assigned groups of Components, which are then processed using Systems.
Components include the contents of the source file, the location of the source file, the derived triples, etc.
Systems are functions that are grouped in schedules, one system might derive defined owl properties from derived triples and might be present in the \textit{Parse} schedule.
Each language can then add their specific systems into the defined schedules for language specific functionality.
This way common systems are defined once, resulting in a consistent experience over all semantic languages.
Bour et al. state "No spec, no tests"\cite{Bour_2018}, meaning it is difficult to write a useful testsuite for language servers.
User-facing features are not well specified and open for interpretation.
At least common systems act the same way over different semantic languages.

\subsection{Langauge Server schedules}

This section goes into detail on the different systems that are implemented for each schedule of the language server.
Some systems only create components, these components are then used by other systems, potentially in different schedules.
If an expected component is not present, the system will skip that entity, allowing for asynchronously running systems.

\subsubsection*{Parse}

\begin{figure}[!ht]
    \centering
    \includegraphics[width=0.95\textwidth]{./images/ParseSchedule.pdf}
    \caption{Visual representation of our example pipeline, 
        loading sensor data from The Things Network into a triple store}\label{fig:Pipeline}
\end{figure}


\subsubsection*{Diagnostics}




\subsubsection*{SemanticHighlighting}

\subsubsection*{Completion}



\section{Usage and Potential Impact Across User Groups}
\label{sec:usage}

This section discusses how SWLS can be used and the expected impact it may have across different user groups.
As a language server, SWLS provides functionality that is independent of specific editors, making it highly versatile.
Some editors, such as NeoVim\footnote{\url{https://neovim.io/}}, treat language servers as first-class citizens, offering seamless integration.
Others, like Visual Studio Code\footnote{\url{https://code.visualstudio.com/}} and Monaco\footnote{\url{https://microsoft.github.io/monaco-editor/}}, require additional glue code to enable effective interaction.
Regardless of the editor, SWLS delivers a consistent set of features that enhance the user experience across all supported platforms.

All code is under a MIT licence available on github\footnote{\url{https://github.com/ajuvercr/semantic-web-lsp}}.
There are currently three ways to interact with SWLS:
\begin{enumerate}
  \item Download the VSCode extension from the market place\footnote{\url{https://marketplace.visualstudio.com/items?itemName=ajuvercr.semantic-web-lsp}}.
  \item Download and compile the binary locally and integrate it into NeoVim\footnote{Installation instruction are available in the README \url{https://github.com/ajuvercr/semantic-web-lsp}}.
  \item Go to the demo available online\footnote{\url{https://ajuvercr.github.io/semantic-web-lsp/}}.
\end{enumerate}

In the demo, users are presented with four distinct editors, each tailored to a specific use case.
A toy ontology and its accompanying SHACL shapes serve as foundational resources for the other editors.
One editor is designed for creating a small dataset based on the toy ontology.
Another editor showcases a simple SPARQL query, which, when executed using a SPARQL client, could provide insights from data structured according to the ontology.
This setup allows users to immerse themselves in different personas, gaining hands-on experience with various aspects of semantic data workflows.

While all editors, except for the SPARQL editor, are functionally the same, 
the visual seperation improves the realism of the tasks at hand.
Below, we discuss how SWLS benefits each user group and validate its utility in practice.
We first go over the key ways the SWLS might help users, then we will apply these ways to the different user groups defined in Section \ref{sec:introduction}, and how those users groups can be assisted by the SWLS.

\subsection{Key Improvements by SWLS}

\begin{figure}[tb]
    \centering
    \begin{subfigure}{0.48\textwidth}
      \includegraphics[width=\textwidth]{./images/hover.png}
      \caption{SWLS shows the user type information on hover as well as diagnostics}
      \label{hover}
    \end{subfigure}
    \hfill
    \begin{subfigure}{0.48\textwidth}
      \includegraphics[width=\textwidth]{./images/class.png}
      \caption{SWLS completes a class when the user wants to write a class}
      \label{class_completion}
    \end{subfigure}
    \hfill
    \begin{subfigure}{0.48\textwidth}
      \includegraphics[width=\textwidth]{./images/property.png}
      \caption{SWLS completes properties, first the properties with the correct domain}
      \label{property_completion}
    \end{subfigure}
    \hfill
    \begin{subfigure}{0.48\textwidth}
      \includegraphics[width=\textwidth]{./images/undefined.png}
      \caption{SWLS notifies the user of undefined prefixes}
      \label{undefined_prefix}
    \end{subfigure}
    \caption{
      Demo application that shows the usage of SWLS. Features include hover information, autocompletion and diagnostics.
    }\label{lst:Demo}
\end{figure}

SWLS enhances development efficiency by providing robust autocompletion capabilities, as illustrated in Figures \ref{class_completion} and \ref{property_completion}.
These features facilitate the creation of semantic documents by accelerating the authoring process and reducing the likelihood of errors.
Whether defining ontologies, constructing SHACL shapes, or formulating SPARQL queries, context-aware suggestions for relevant classes and properties minimize repetitive tasks and cognitive effort, allowing users to concentrate on the conceptual aspects of their work.
Additionally, autocompletion assists users in exploring the current domain by offering informed suggestions based on contextual information.

Beyond efficiency, SWLS enhances comprehension by delivering real-time feedback on semantic data.
It improves users' confidence in both syntax and semantics by identifying issues such as undefined prefixes (Figure \ref{undefined_prefix}) and SHACL constraint violations (Figure \ref{hover}).
The hover feature (Figure \ref{hover}) provides detailed information, including associated classes, enabling users to quickly understand the structure and relationships within a document.
This immediate access to contextual insights and validation mechanisms reduces reliance on external references and serves as an educational aid for mastering Semantic Web technologies.


\subsection{User groups}

\paragraph{Power users,} who frequently interact with Semantic Web technologies, rely on their deep understanding of common properties and domain-specific knowledge to navigate tasks efficiently.
SWLS enhances their workflows by providing robust autocompletion features that streamline document creation and reduce repetitive tasks.
Additionally, the language server supports quality assurance by helping power users identify invalid entities (using shape validation), ensuring their semantic documents are both accurate and high quality.

\paragraph{For newcomers to the Semantic Web,} engaging with these technologies can be daunting due to challenges with syntax, semantics, and validation. 
SWLS addresses these pain points by providing immediate feedback on syntax errors, which helps users learn the correct structure and lessens frustration.
The autocompletion feature offers guided support by suggesting relevant properties, allowing users to build confidence in creating semantic documents.
Moreover, validation tools play an educational role: newcomers can intentionally trigger validation errors to understand what constitutes a faulty document, turning mistakes into valuable learning opportunities.

\paragraph{Domain experts,} a specialized subset of power users, focus on assessing and curating domain-specific ontologies.
By hovering over properties and classes (Figure \ref{hover}), domain experts can quickly verify descriptions and relationships, ensuring the ontology aligns with their domain knowledge.
Autocompletion further aids in this process by completing relevant terms, helping experts assess whether the ontology feels correct in practice.

\paragraph{Data engineers,} would primarily work with SPARQL queries to extract meaningful insights from semantic data. 
For these users, SWLS provides support by offering autocompletion for classes and properties, as well as syntax validation to prevent errors during query formulation. 
These features significantly enhance productivity by reducing the time spent debugging queries and ensuring accurate results. 
By streamlining the querying process, SWLS enables data engineers to focus on deriving insights rather than resolving technical challenges.



\section{Conclusion}%
\label{sec:conclusion}
% \info{Something missing for me is a clear list of what is supported and what is not. From the text I understand that Turtle and JSON-LD are supported for RDF, but not N-Triples, RDF/XML, Trig, etc. Perhaps mention it either in the Introduction or in Section 3 and reiterate here?}

% In this paper we introduced SWLS to bridge the gap between semantic technologies and their practical application by offering an extensible and editor-agnostic tool tailored to diverse user groups.
% By providing autocompletion, validation, highlighting, and contextual documentation, SWLS aims on enhancing productivity, improving the quality of semantic documents, and lowering the entry barrier for newcomers.
% Its integration of foundational editors for creating ontologies, designing SHACL shapes, writing SPARQL queries, and authoring sample data allows users to experience the full spectrum of semantic workflows in an intuitive and accessible environment.
%
% Through its ability to streamline domain exploration, improve semantic understanding, and foster confidence in document quality, SWLS demonstrates its potential to support both beginners and seasoned experts in Semantic Web technologies.
% Its modular design, which builds upon existing libraries such as \texttt{rudof}, ensures that it is not only a powerful standalone tool but also a platform capable of evolving with advancements in the field.
%
% Thanks to the ECS design pattern, SWLS has an extensible architecture ensuring that new features and improvements can be integrated.
% SWLS is in this way ready for developers to tackle more specialized scenarios.
% This could include for example, writing configuration files for the RDF-based dependency injection framework ComponentsJS (CJS)~\cite{01GPAWNQ5ZS2DAY0J9JMPQHM9C} or authoring YARRRML knowledge graph construction files~\cite{Heyvaert2018Declarative}.
%
% For example, users working with CJS configurations often encounter challenges when validating their configurations, frequently resorting to runtime checks for correctness.
% Standard JSON-LD editors struggle with CJS configurations due to their reliance on Linked Software Dependencies~\cite{CJS2}, which need to be resolved against local \texttt{node\_modules} directories.
% By extending SWLS with the following capabilities, CJS users could benefit from the same powerful IDE support already available to Semantic Web developers by: (i) implementing a mechanism to resolve JSON-LD contexts by referencing configuration files in \texttt{node\_modules}; (ii)deriving properties directly from CJS component definitions to streamline autocompletion; and (iii) validating configurations by ensuring all parameters are defined and conform to the correct data types and ranges.
% % \todo{PC - I would leave this out. Update JR: Is a bit too detailed but not harmful IMO. Remove it if struggling for space.}
%
% Beyond these specialized enhancements, SWLS can also be improved by refining its existing systems.
% For example, the current type inference mechanism relies solely on the \texttt{rdf:type} predicate to determine entity types.
% Expanding this capability with reasoning support would enhance the accuracy of autocompletions and hover information, making the language server even more effective for users.
%
% The SWLS can be further enhanced by expanding its support for additional semantic languages.
% For instance, incorporating Trig support would require minimal effort, as it mainly involves extending the existing Turtle tokenizer and parser.
% Once this extension is in place, the majority of SWLS’s current functionality would work seamlessly with TRiG.
%
% In contrast, adding support for a language like N3 presents a more significant challenge.
% N3’s parser is notably more complex than simpler formats such as Turtle, and users of SWLS would naturally expect the editor to support basic reasoning tasks inherent to the N3 language.
% Implementing such reasoning capabilities opens a vast field of possibilities, limited only by the scope of future development efforts.
%
% However, SWLS does not exist in isolation but benefits from a collaborative ecosystem.
% Just as SWLS leverages the \texttt{rudof} crate for SHACL and ShEx validation, it is anticipated that other Semantic Web libraries will emerge.
% The design of SWLS should ensure that these new tools can be easily integrated, enabling the system to evolve alongside the broader Semantic Web landscape.
% By building on existing tools and fostering compatibility with new ones, SWLS can continue to meet the diverse needs of its users and maintain its relevance in this rapidly advancing field.
%

In this paper, we introduced SWLS as an extensible, editor-agnostic tool designed to enhance productivity and accessibility in Semantic Web technologies. By providing autocompletion, validation, highlighting, and contextual documentation, SWLS streamlines ontology creation, SHACL shape design, SPARQL query formulation, and sample data authoring, benefiting both newcomers and experienced users.

The modular architecture, built on the ECS pattern and leveraging 	exttt{rudof}, ensures that SWLS can evolve alongside Semantic Web advancements. This extensibility enables specialized applications, such as improving validation for ComponentsJS~\cite{01GPAWNQ5ZS2DAY0J9JMPQHM9C} configurations or supporting YARRRML knowledge graph construction~\cite{Heyvaert2018Declarative}, reducing the need for runtime checks and manual corrections.

Future enhancements include refining type inference with reasoning support, improving autocompletions and hover information. Expanding support for additional semantic languages, such as Trig and N3, presents varying degrees of complexity, with N3 requiring deeper integration of reasoning capabilities.

SWLS does not exist in isolation but as part of a collaborative ecosystem. By maintaining compatibility with emerging Semantic Web libraries, it ensures continued relevance and adaptability. Its extensible design enables seamless integration of new tools, fostering innovation and supporting the evolving needs of the Semantic Web community.



% \begin{credits}
% \subsubsection{\ackname} A bold run-in heading in small font size at the end of the paper is
% used for general acknowledgments, for example: This study was funded
% by X (grant number Y).
%
% \subsubsection{\discintname}
% It is now necessary to declare any competing interests or to specifically
% state that the authors have no competing interests. Please place the
% statement with a bold run-in heading in small font size beneath the
% (optional) acknowledgments\footnote{If EquinOCS, our proceedings submission
% system, is used, then the disclaimer can be provided directly in the system.},
% for example: The authors have no competing interests to declare that are
% relevant to the content of this article. Or: Author A has received research
% grants from Company W. Author B has received a speaker honorarium from
% Company X and owns stock in Company Y. Author C is a member of committee Z.
% \end{credits}

\bibliographystyle{splncs04}
\bibliography{bibliography}

\end{document}
