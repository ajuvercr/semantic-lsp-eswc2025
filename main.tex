\documentclass[runningheads]{llncs}
\usepackage[T1]{fontenc}
\usepackage{algpseudocode, algorithm, amsmath, amssymb}					% packages to get the fonts, symbols used in most math
%\usepackage{dcolumn}
\usepackage[english]{babel}
\usepackage{hyperref}
\usepackage{multirow}
\usepackage{latexsym}
\usepackage{listings}
\usepackage{booktabs}
% T1 fonts will be used to generate the final print and online PDFs,
% so please use T1 fonts in your manuscript whenever possible.
% Other font encondings may result in incorrect characters.
%
\usepackage{graphicx}
% Used for displaying a sample figure. If possible, figure files should
% be included in EPS format.
%
\usepackage[dvipsnames,svgnames, table]{xcolor} % text color
\DeclareFontFamily{U}{mathx}{\hyphenchar\font45}
\DeclareFontShape{U}{mathx}{m}{n}{
      <5> <6> <7> <8> <9> <10>
      <10.95> <12> <14.4> <17.28> <20.74> <24.88>
      mathx10
      }{}
\DeclareSymbolFont{mathx}{U}{mathx}{m}{n}
\DeclareMathSymbol{\bigtimes}{1}{mathx}{"91}



\DeclareRobustCommand{\ojoin}{\rule[0.10ex]{.3em}{.4pt}\llap{\rule[1.40ex]{.3em}{.4pt}}}
\newcommand{\leftouterjoin}{\mathrel{\ojoin\mkern-6.5mu\Join}}
\newcommand{\rightouterjoin}{\mathrel{\Join\mkern-6.5mu\ojoin}}
\newcommand{\fullouterjoin}{\mathrel{\ojoin\mkern-6.5mu\Join\mkern-6.5mu\ojoin}}


\definecolor{eclipseStrings}{RGB}{42,0.0,255}
\definecolor{eclipseKeywords}{RGB}{127,0,85}
\definecolor{darkgreen}{RGB}{0,128,21}
\colorlet{numb}{magenta!60!black}
\setlength{\parskip}{0pt}

\algrenewcommand\algorithmicrequire{\textbf{Input:}}
\algrenewcommand\algorithmicensure{\textbf{Returns:}}


\lstdefinelanguage{sparql}{
    basicstyle=\tiny,
    keywordstyle=\color{eclipseKeywords},
    commentstyle=\color{blue}, % style of comment
    stringstyle=\color{darkgreen},
    keywords={FILTER, NOT, EXISTS, bound, OPTIONAL, BIND, IF, isIRI, AS, INSERT, UPDATE, DELETE, WHERE, SELECT, UNION, rr,rml,ql,rdfs},
    morecomment=[n]{?}{\ },
    morecomment=[l]\#,
    morestring=[b]",
    morestring=[s]{<}{>},
    frame=lines,
}
\lstdefinelanguage{rdf}{
    basicstyle=\tiny,
    keywordstyle=\color{eclipseKeywords},
    commentstyle=\color{gray}, % style of comment
    stringstyle=\color{darkgreen},
    keywords={rml, ql, rr, rdfs},
    morecomment=[n]{?}{\ },
    morecomment=[l]\#,
    morestring=[b]",
    morestring=[s]{<}{>},
    frame=lines,
}


\lstdefinelanguage{triples}{
    basicstyle=\tiny,
    keywordstyle=\color{eclipseKeywords},
    commentstyle=\color{darkgreen}, % style of comment
    keywords={a, rml, ql, rr, rdfs},
    morecomment=[s]{?}{\ },
    frame=lines,
}

  

% Set symbols % 
\newcommand{\symFragmentUniverse}{\mathcal{F}}
\newcommand{\symAttrUniverse}{\mathcal{A}}
\newcommand{\symTermUniverse}{\mathcal{T}}
\newcommand{\symAllStrings}{\mathcal{S}} % the symbol for the set of all strings
\newcommand{\symAllIRIs}{\mathcal{I}} % the symbol for the set of all IRIs
\newcommand{\symAllLiterals}{\mathcal{L}} % the symbol for the set of all literals
\newcommand{\symAllBNodes}{\mathcal{B}} % the symbol for the set of all blank nodes
\newcommand{\symRMLGraph}{G}
\newcommand{\symDataObjUni}{\mathcal{D}}
\newcommand{\symDataAccUni}{\mathcal{Q}}
\newcommand{\symDataSeqUni}{\bar{\mathcal{O}}}
\newcommand{\symAttrSubSet}{A}
\newcommand{\symProjSet}{P}
\newcommand{\symMappingTupleSet}{M}
\newcommand{\symRenamePFunc}{R}
\newcommand{\bigDataObject}{D}
\newcommand{\symSetDataset}{\mathcal{D}^\texttt{\tiny ds}\!}
\newcommand{\symSetDatasetX}[1]{\mathcal{D}^\texttt{\tiny ds}_{#1}}
\newcommand{\symSetContentA}{\mathcal{D}^\texttt{\tiny c1}}
\newcommand{\symSetContentAX}[1]{\mathcal{D}^\texttt{\tiny c1}_{#1}}
\newcommand{\symSetContentB}{\mathcal{D}^\texttt{\tiny c2}}
\newcommand{\symSetContentBX}[1]{\mathcal{D}^\texttt{\tiny c2}_{#1}}
\newcommand{\symDataAcc}{L}
\newcommand{\symDataAccX}[1]{L_{#1}}
\newcommand{\symDataSeqCard}[1]{D^{seq,#1}}
\newcommand{\symAttrQueryMap}{\mathbb{P}}
\newcommand{\symDataSeq}{\bar{O}}
\newcommand{\symExtExprSet}{E}
\newcommand{\symExtExprSeq}{\bar{\symExtExprSet}}
\newcommand{\symExtPairs}{\symExtExprSet}
\newcommand{\symTermSeqUniverse}{\bar{\symTermUniverse}}
\newcommand{\symFragmentPFunc}{\bbbf}
\newcommand{\symFragmentSet}{F}
\newcommand{\symJoinAttrPairs}{\mathbb{J}}
\newcommand{\symExtFuncUni}{\mathcal{E}}

% Special values % 
\newcommand{\card}{card}
\newcommand{\fragAttr}{a_\mathrm{frag}}
\newcommand{\fragDefault}{f_{0}}
\newcommand{\attr}{a}
\newcommand{\subjAttr}{\attr_\textrm{s}}
\newcommand{\predAttr}{\attr_\textrm{p}}
\newcommand{\objAttr}{\attr_\textrm{o}}
\newcommand{\graphAttr}{\attr_\textrm{g}}
\newcommand{\fragment}[0]{fragment}
\newcommand{\error}{\epsilon}
\newcommand{\mappingTuple}{t}
\newcommand{\symMappingInst}{I}
\newcommand{\mappingRel}{(\symAttrSubSet, \symMappingInst)}
\newcommand{\mappingRelSpec}[1]{(\symAttrSubSet_{#1}, \symMappingInst_{#1})}
\newcommand{\queryExpr}{q}
\newcommand{\rootIt}{rit}
\newcommand{\cobit}{cobit}
\newcommand{\dataObject}{d}
\newcommand{\eval}{\mathit{eval}}
\newcommand{\evalX}[1]{\eval_{#1}}
\newcommand{\cbeval}{\mathit{eval}'\!}
\newcommand{\cbevalX}[1]{\eval_{#1}'}
\newcommand{\sourceType}{type}
\newcommand{\sourceTypeX}[1]{type_{#1}}
\newcommand{\sourceTypeTuple}{(\symSetDataset, \symSetContentA\!, \symSetContentB\!, \symDataAcc, \symDataAcc'\!, \eval, \cbeval, \typeCast)}
\newcommand{\sourceTypeTupleX}[1]{\sourceTypeTupleXXX{#1}{#1}{#1}}
\newcommand{\sourceTypeTupleXXX}[3]{(\symSetDatasetX{#1}, \symSetContentAX{#2}, \symSetContentBX{#3}, \symDataAccX{#2}, \symDataAccX{#3}', \evalX{#2}, \cbevalX{#3}, \typeCastX{#1})}
\newcommand{\dataSource}{s}
\newcommand{\dataSourceTuple}{(\sourceType,\bigDataObject)}
\newcommand{\iterConfig}{\bbbc}
\newcommand{\sourceOp}[3]{\textrm{Source}^{(#1,#2,#3)}}
\newcommand{\sourceOpDflt}{\sourceOp{\dataSource}{\queryExpr}{\symAttrQueryMap}}
\newcommand{\sourceOpTuple}{(\dataSource, \iterConfig)}
\newcommand{\attrQueryPair}{(\attr \rightarrow \queryExpr)}
\newcommand{\typeCast}{cast}
\newcommand{\typeCastX}[1]{\typeCast_{#1}}
\newcommand{\length}{len}
\newcommand{\var}{v}
\newcommand{\indexSeq}{M}
\newcommand{\extExpr}{\varphi}
\newcommand{\extEval}{eval}
\newcommand{\extPairsType}{(\symAttrUniverse \setminus \symAttrSubSet) \times \symExtExprSeqUni}
\newcommand{\serdeAttr}{a_{serialized}}
\newcommand{\bgp}{\bbbb}
\newcommand{\extFunc}{f}
\newcommand{\extFuncType}{(\symTermUniverse \cup \{\error\})}
\newcommand{\extFuncTuple}{(\extFunc,\extExpr_1,\dots,\extExpr_n)}
\newcommand{\extOp}{\text{Extend}_{\varphi}^{\attr}}
\newcommand{\extOpAttr}[1]{\text{Extend}_{\varphi}^{#1}}
\newcommand{\renameOp}{\text{Rename}^{\symRenamePFunc}}
\newcommand{\equiJoinOp}{\text{Join}^{\symJoinAttrPairs}_{=}}
\newcommand{\triplesMapIRI}{u}
\newcommand{\triplesMapGraph}[2]{\symRMLGraph_{#1}^{#2}}
\newcommand{\toIRI}{\texttt{toIRI}}
\newcommand{\toBNode}{\texttt{toBNode}^{\LTB}}
\newcommand{\toLiteral}{\texttt{toLiteral}}
\newcommand{\template}{\texttt{template}^{\symAttrQueryMap}}
\newcommand{\initSrc}{\texttt{initSrc}}
\newcommand{\concat}{\texttt{concat}}
\newcommand{\concatSeq}{\texttt{concatSeq}}
\newcommand{\literal}{\ell}
\newcommand{\lex}{\mathit{lex}}
\newcommand{\dt}{\mathit{dt}}
\newcommand{\literalTuple}{(\lex, \dt)}
\newcommand{\LTB}{\mathit{L2B}}
\newcommand{\tempSubStrs}{\texttt{tempSubStrs}}
\newcommand{\substrSeq}{\bar{S}}
\newcommand{\projectOp}[1]{\text{Project}^{#1}}
\newcommand{\natJoin}{\text{NatJoin}}
\newcommand{\union}{\text{Union}}

% Functions % 
\newcommand{\fctDom}[1]{\mathrm{dom}(#1)}
\newcommand{\fctMappingTuple}[1]{\mappingTuple(#1)}
\newcommand{\fctAttrs}[1]{\mathrm{attrs}(#1)}
\newcommand{\fctCard}[2]{\card[#1](#2)}
\newcommand{\fctRootIt}[2]{\rootIt(#1, #2)}
\newcommand{\fctCobit}[3]{\cobit(#1, #2, #3)}
\newcommand{\fctEval}[2]{\eval(#1,#2)}
\newcommand{\fctEvalLang}[3]{\eval^{#3}(#1,#2)}
\newcommand{\fctCbeval}[3]{\cbeval(#1,#2,#3)}
\newcommand{\fctCbevalLang}[4]{\cbeval^{#4}(#1,#2,#3)}
\newcommand{\fctTypeCast}[1]{\typeCast(#1)}
\newcommand{\fctLength}[1]{\length(#1)}
\newcommand{\fctRenameTuple}[2]{rename^{#1}(#2)}
\newcommand{\fctSubst}[2]{subst^{#1}(#2)}
\newcommand{\fctExtFuncEvalN}[2]{\extFunc(\fctEval{#1_1}{#2},\dots,\fctEval{#1_n}{#2})}
\newcommand{\fctExtOp}[1]{\extOp(#1)}
\newcommand{\fctExtOpBig}[1]{\extOp\bigl(#1\bigr)}
\newcommand{\fctExtOpX}[3]{\text{Extend}^{#1}_{#2}(#3)}
\newcommand{\fctExtOpXBig}[3]{\text{Extend}^{#1}_{#2}\bigl(#3\bigr)}
\newcommand{\fctRenameOp}[1]{\renameOp(#1)}
\newcommand{\fctRenameOpBig}[1]{\renameOp\bigl(#1\bigr)}
\newcommand{\fctEquiJoinOp}[2]{\equiJoinOp(#1, #2)}
\newcommand{\fctEquiJoinOpBig}[2]{\equiJoinOp\bigl(#1, #2\bigr)}
\newcommand{\fctToIRI}[1]{\toIRI(#1)}
\newcommand{\fctToBNode}[1]{\toBNode(#1)}
\newcommand{\fctToLiteral}[1]{\toLiteral(#1)}
\newcommand{\fctTemplate}[1]{\template(#1)}
\newcommand{\fctImg}[1]{\mathrm{img}(#1)}
\newcommand{\fctInitSrc}[1]{\initSrc(#1)}
\newcommand{\fctConcat}[2]{\concat(#1, #2)}
\newcommand{\fctConcatSeq}[1]{\concatSeq(#1)}
\newcommand{\fctTempSubStrs}[1]{\tempSubStrs(#1)}
\newcommand{\fctProjectOpDflt}[1]{\fctProjectOp{\symProjSet}{#1}}
\newcommand{\fctProjectOp}[2]{\projectOp{#1}(#2)}
\newcommand{\fctProjectOpBig}[2]{\projectOp{#1}\bigl(#2\bigr)}
\newcommand{\fctNatJoin}[2]{\natJoin(#1,#2)}
\newcommand{\fctUnion}[2]{\union(#1,#2)}

% Layout %
\newcommand{\ttl}[1]{\texttt{\small #1}}  % for conrete IRIs, etc. used in examples and in formulas



%%%%%%%%%%%%%%%%%%%%%%%%%%%%%%%%%%%%%%%%%%%%%%%
% The following ensures that we have a (non-visible) table of contents embedded
% in the PDF, which PDF readers can show and, thus, allows me to navigate the
% document more easily.
%                                    Olaf
\setcounter{tocdepth}{2}
\hypersetup{bookmarksopen=true, citecolor=blue}
%
%%%%%%%%%%%%%%%%%%%%%%%%%%%%%%%%%%%%%%%%%%%%%%%


\begin{document}												% end of preamble and beginning of text that will be printed

\title{Semantic Languages servers are cool}
%
%\titlerunning{Abbreviated paper title}
% If the paper title is too long for the running head, you can set
% an abbreviated paper title here
%
\author{Arthur Vercruysse\inst{1}\orcidID{0000-0000-0000-0000} \and
Julian\inst{1}\orcidID{0000-0002-1741-2090} \and 
Pieter Colpaert\inst{1}\orcidID{0000-0002-1741-2090}}
%

\authorrunning{A. Vercruysse et al.}
% First names are abbreviated in the running head.
% If there are more than two authors, 'et al.' is used.
%
\institute{
University of Ghent - imec, Ghent, Belgium \\
\email{arthur.vercruysse@ugent.be}}
%
\maketitle              % typeset the header of the contribution
%
% \begin{abstract}
% % Context (What is needed to understand the "need"?)
%   The semantic web has produced many syntaxes to interact with it: from data format to querying to reasoning.
% % Need (Why something needed to be done at all?)
%   These formats all suffer from human imprecisions, a single typo changes the entire semantics of the document, leaving it non-interoperable.
% % Task (What was undertaken to address the need? It’s here that you write ‘in this paper, we …’)
%   In this paper, we introduce the semantic language server.
% % Object (What the present document does or covers)
%   The language server enhances semantic documents with IDE functionality. 
%   Notifying the users early about potential mistakes from typo's to SHACL violations.
% % Findings (What the work done yielded or revealed)
%   By combining and extending the state of the art, we enhanced the efficienty, precision and confidance of end users working with semantic documents,
%   including power users, newcomers, domain experts and data engineers.
% % Conclusion (What the findings mean for the audience)
%   With the semantic language server users can expect the similar functionality as existing tools like Yasgui, but closer the user in their coding environment.
% % Perspectives (What the future holds, beyond this work)
% \keywords{Language Server, IDE, Tool}
% \end{abstract}


\begin{abstract}
% Context (What is needed to understand the "need"?)
The semantic web has introduced a variety of syntaxes for data representation, querying, and reasoning, such as Turtle, SPARQL, and SHACL.
% Need (Why something needed to be done at all?)
While these formats enable powerful interactions with linked data, they are highly sensitive to human errors; even minor typos can disrupt the semantics of a document, rendering it invalid or non-interoperable.
% Task (What was undertaken to address the need? It’s here that you write ‘in this paper, we …’)
In this paper, we present the Semantic Language Server, a tool designed to enhance the editing experience for semantic web documents by integrating IDE functionalities directly into the user's coding environment.
% Object (What the present document does or covers)
The language server provides features such as real-time syntax validation, autocompletion, and SHACL-based diagnostics to notify users of potential mistakes early in the development process.
The language server is available as extensions for popular platforms like VS Code and Neovim, as well as in a standalone web-based interface (integrated into a Monaco editor).
% Findings (What the work done yielded or revealed)
By building on and extending the state of the art, our tool improves the efficiency, precision, and confidence of various user groups, including semantic power users, newcomers, domain experts, and data engineers.
% Conclusion (What the findings mean for the audience)
Semantic Language Server offers comparable functionality to existing tools like YASGUI but with greater accessibility and integration into familiar development workflows. 
% Perspectives (What the future holds, beyond this work)
This work not only addresses current limitations in semantic web tooling but also paves the way for broader adoption of semantic technologies by reducing barriers and improving usability.
\keywords{Language Server, IDE, Tool}
\end{abstract}



% \begin{credits}
% \subsubsection{\ackname} A bold run-in heading in small font size at the end of the paper is
% used for general acknowledgments, for example: This study was funded
% by X (grant number Y).
%
% \subsubsection{\discintname}
% It is now necessary to declare any competing interests or to specifically
% state that the authors have no competing interests. Please place the
% statement with a bold run-in heading in small font size beneath the
% (optional) acknowledgments\footnote{If EquinOCS, our proceedings submission
% system, is used, then the disclaimer can be provided directly in the system.},
% for example: The authors have no competing interests to declare that are
% relevant to the content of this article. Or: Author A has received research
% grants from Company W. Author B has received a speaker honorarium from
% Company X and owns stock in Company Y. Author C is a member of committee Z.
% \end{credits}

\bibliographystyle{splncs04}
\bibliography{bibliography}

\end{document}
