% \begin{abstract}
% % Context (What is needed to understand the "need"?)
%   The semantic web has produced many syntaxes to interact with it: from data format to querying to reasoning.
% % Need (Why something needed to be done at all?)
%   These formats all suffer from human imprecisions, a single typo changes the entire semantics of the document, leaving it non-interoperable.
% % Task (What was undertaken to address the need? It’s here that you write ‘in this paper, we …’)
%   In this paper, we introduce the semantic language server.
% % Object (What the present document does or covers)
%   The language server enhances semantic documents with IDE functionality. 
%   Notifying the users early about potential mistakes from typo's to SHACL violations.
% % Findings (What the work done yielded or revealed)
%   By combining and extending the state of the art, we enhanced the efficienty, precision and confidance of end users working with semantic documents,
%   including power users, newcomers, domain experts and data engineers.
% % Conclusion (What the findings mean for the audience)
%   With the semantic language server users can expect the similar functionality as existing tools like Yasgui, but closer the user in their coding environment.
% % Perspectives (What the future holds, beyond this work)
% \keywords{Language Server, IDE, Tool}
% \end{abstract}


\begin{abstract}
% Context (What is needed to understand the "need"?)
The semantic web has introduced a variety of syntaxes for data representation, querying, and reasoning, such as Turtle, SPARQL, and SHACL.
% Need (Why something needed to be done at all?)
While these formats enable powerful interactions with linked data, they are highly sensitive to human errors; even minor typos can disrupt the semantics of a document, rendering it invalid or non-interoperable.
% Task (What was undertaken to address the need? It’s here that you write ‘in this paper, we …’)
In this paper, we present the Semantic Language Server, a tool designed to enhance the editing experience for semantic web documents by integrating IDE functionalities directly into the user's coding environment.
% Object (What the present document does or covers)
The language server provides features such as real-time syntax validation, autocompletion, and SHACL-based diagnostics to notify users of potential mistakes early in the development process.
% The language server is available as extensions for popular platforms like VS Code and Neovim, as well as in a standalone web-based interface (integrated into a Monaco editor).

% Findings (What the work done yielded or revealed)
By building on and extending the state of the art, our tool aims on improving the efficiency, precision, and confidence of various user groups, including semantic power users, newcomers, domain experts, and data engineers.
% Conclusion (What the findings mean for the audience)
The Semantic Language Server offers functionality comparable to tools like YASGUI while extending support to a broader range of semantic web tasks and diagnostics. 
It also integrates seamlessly into established development environments such as VS Code, Neovim, and a Monaco (a standalone web editor). 
Its layered architecture facilitates extending the code base to support new features, while following the well known language server protocol allows for easy integration into other environments. 
In future work, we will further extend feature support and perform user evaluations that allow us to identify improvement points. 
% Perspectives (What the future holds, beyond this work)
% This work not only addresses current limitations in semantic web tooling but also paves the way for broader adoption of semantic technologies by reducing barriers and improving usability.
\keywords{Language Server, IDE, Tool}
\end{abstract}
