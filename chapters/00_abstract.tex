% \begin{abstract}
% % Context (What is needed to understand the "need"?)
%   The semantic web has produced many syntaxes to interact with it: from data format to querying to reasoning.
% % Need (Why something needed to be done at all?)
%   These formats all suffer from human imprecisions, a single typo changes the entire semantics of the document, leaving it non-interoperable.
% % Task (What was undertaken to address the need? It’s here that you write ‘in this paper, we …’)
%   In this paper, we introduce the semantic web language server.
% % Object (What the present document does or covers)
%   The language server enhances semantic documents with IDE functionality. 
%   Notifying the users early about potential mistakes from typo's to SHACL violations.
% % Findings (What the work done yielded or revealed)
%   By combining and extending the state of the art, we enhanced the efficienty, precision and confidance of end users working with semantic documents,
%   including power users, newcomers, domain experts and data engineers.
% % Conclusion (What the findings mean for the audience)
%   With the semantic web language server users can expect the similar functionality as existing tools like Yasgui, but closer the user in their coding environment.
% % Perspectives (What the future holds, beyond this work)
% \keywords{Language Server, IDE, Tool}
% \end{abstract}


\begin{abstract}
% Context (What is needed to understand the "need"?)
The semantic web has introduced a variety of syntaxes for e.g., serializing, querying, and validating linked data, such as Turtle, SPARQL, and SHACL.
% Need (Why something needed to be done at all?)
While these formats enable powerful interactions with Linked Data, they are highly sensitive to human mistakes; even minor typos can disrupt the semantics of a document, rendering it invalid or non-interoperable.
% Task (What was undertaken to address the need? It’s here that you write ‘in this paper, we …’)
In this paper, we study how the the authoring experience of Semantic Web documents can be enhanced through the use of the Language Server Protocol with for instance code completion, syntax highlighting and live validation output.
% Object (What the present document does or covers)
To that extent, we introduce the Semantic Web Language Server (SWLS), a language server with features such as real-time syntax validation, autocompletion, and SHACL-based diagnostics to notify users of potential mistakes early in the development process.
% The language server is available as extensions for popular platforms like VS Code and Neovim, as well as in a standalone web-based interface (integrated into a Monaco editor).
% Findings (What the work done yielded or revealed)
By extending functionalities beyond what is already supported by the best-in-class YAGUI interface, our tool aims to further improve the development efficiency, precision, and confidence of semantic power users, newcomers, domain experts, and data engineers.
% Conclusion (What the findings mean for the audience)
It integrates seamlessly into established Web-based and desktop development environments, and its layered architecture allows extending the code base to support new features in the future. 
% PC: This is a main track resource paper, not a workshop paper. Don’t promise an evaluation in future work.
% In future work, we will further extend feature support and perform user evaluations that allow us to identify improvement points. 
% Perspectives (What the future holds, beyond this work)
% This work not only addresses current limitations in semantic web tooling but also paves the way for broader adoption of semantic technologies by reducing barriers and improving usability.
\keywords{Language Server, IDE, Tool, EUSE, Semantic Web}
\end{abstract}

\textbf{Resource Types:} An open-source implementation and a Web-based demonstrator

\textbf{URL:} \url{https://github.com/ajuvercr/jsonld-lsp} 

\textbf{Licenses:} MIT License

\textbf{DOI:} [TODO add DOI] 
