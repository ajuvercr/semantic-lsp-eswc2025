\section{Introduction}%
\label{sec:introduction}

% Enhancing semantic web adoption by lowering the barrier to learn (ref Two Decades On)
% In this paper we introduce Semantic LSP, a language server to bring enhanced editing linked data documents, including turtle, jsonld and sparql.

% Focus not only on semantic engineers, people that work daily with semantic data
% But also focus on data engineers that create intriguing sparql queries over the precious semantic data
% Focus also on the domain experts, sometimes installing extra software is too big a hurdle
% These people we call 'end users' (cite  10.1145/1922649.1922658)

% End users should focus on the following steps
% - Requirements: How the software should behave in the world
% - Design and specifications: How the software behaves internally to achieve the requirements
% - Reuse: Using preexisting code to save time and avoid errors (including integration, extension, and other perfective maintenance)
% - Testing and Verification: Gaining confidence about correctness and identifying failures.
% - Debugging: Repairing known failures by locating and correcting errors.
% Language servers help the end user with design and reuse by bringing autocompletion to the editor. 
% Design is more fluent for the domain experts because writing OWL and RDFS is faster and less error prone.
% Reuse is stimulated by already suggesting defined properties when writing turtle.
% Testing and verification is enhanced by using SHACL shapes to valide user data on the fly and ensuring correct syntax.

% A language server also serves to lowering the barrier to active learning. Users in the study ... et al attested to the complexity of creating sparql queryies for wikidata.
% Autocompletion on properties and suggesting properties with plain texts help the students explore the ontology and lets them create more intriguing queries.


% There is already a lot of related work, even in the area of the semantic web (cite yasgui).
% Yasgui showed that lowering the barrier to entry indeed does encourage more usage and raised the semantic literacie quite quickly.
% We believe that taking those concepts and expanding on them to support more semantic languages and focussing on a easy to use design that promotes extensibily is a great oportunity to advance the state of the art.

The Semantic Web~\cite{Berners-Lee_SA_2001} has matured over the past two decades, yet its broad adoption remains a challenge.
One of main critiques appears to be the usability of the technology and its accessibility to newcomers~\cite{10.3233/SW-190387}.
It is difficult for users across varying levels of expertise to effectively create, manage, and query Semantic data, which we attribute mainly to a lack of proper tooling.
This need becomes evident when considering distinct challenges faced by some key user groups:

\begin{enumerate}
  \item \textbf{Power Users}\\
    Experienced Semantic engineers require advanced tools to efficiently manage large datasets, develop and maintain ontologies, and perform reasoning tasks.
    While proficient, even these users benefit from features that reduce cognitive overhead, minimize manual errors, and accelerate routine operations.
    % PC: Next sentence commented: too early - this is not a sales pitch!
    %Tools like Semantic LSP provide real-time feedback, context-aware autocompletion, and error diagnostics, significantly streamlining their workflows.

  \item \textbf{Newcomers and Casual Users}\\
    Beginners, including students, hobbyists and sofware developers who didn't encounter semantic technologies before, often struggle with its intricacies~\cite{EvensteinSigalov2023,Turki2021RepresentingCI}. 
    Writing valid RDF documents or crafting correct SPARQL queries can be overwhelming without proper guidance. 
    % PC: too early
    %Semantic LSP supports active learning by offering user-friendly features such as syntax highlighting, contextual autocompletion, and hover-based documentation, empowering newcomers to experiment confidently and overcome the steep learning curve.

\item \textbf{Domain Experts}\\
    Domain experts often need to interact with Semantic technologies to assess and curate domain-specific data models, typically relying on tools like Protégé. 
    However, such tools can introduce bottlenecks, particularly for smaller projects, due to setup complexity and steep learning curves. 
    %Semantic LSP offers a lightweight alternative by enabling ontology creation directly within text editors using Turtle.
    %This approach is especially beneficial for quick iterations and avoids the overhead of additional software installation and configuration.

  \item \textbf{Data Engineers}\\
    Data engineers frequently face challenges in exploring and extracting data from RDF graphs based on unfamiliar ontologies due to the potentialy complex and abstract nature of inherent Semantic data models.
    Writing complex SPARQL queries over such data may lead to delays in development processes and inneficient data analyses and reuse. 
    %Semantic LSP facilitates these tasks with intelligent autocompletion, which suggests properties and classes based on context, and hover-based explanations that help users understand ontology structures.
    %These features reduce errors, improve query quality, and promote deeper engagement with semantic data.
\end{enumerate}

Integrated Development Environments (IDEs) are widely recognized for their ability to improve developer productivity, streamline workflows, and reduce common mistakes across various domains~\cite{javaEngineer}. 
Features like syntax highlighting and diagnostics are not merely conveniences; they are essential tools for ensuring accuracy and maintaining consistency. 
In recent years, the Language Server Protocol has become a de-facto standard\footnote{\url{https://microsoft.github.io/language-server-protocol/}}, supported in most IDEs to provide such features in an IDE-agnostic way.
Thanks to LSPs, Visual Studio Code, Neovim or event the web-based Monaco editor for example, can be extended with language-specific functionalities by reusing the same Language Server implementation.
%Despite these benefits, such productivity-enhancing features are underrepresented in semantic web tooling.

To address the challenges commonly faced by the aforementioned key user groups, we introduce the Semantic Web Language Server (SWLS).
This is a language server designed to enhance the editing experience specifically for Semantic Web-related documents. 
For power users, SWLS intends to provide real-time feedback, context-aware autocompletion, and error diagnostics, aiming to facilitate and streamline their workflows.
For newcomers, SWLS aims to supports active learning by offering user-friendly features such as syntax highlighting, contextual autocompletion and hover-based documentation, empowering newcomers to experiment confidently and overcome the steep learning curve.
For domain experts, SWLS aims to offer a lightweight alternative to for example Protégé, by enabling ontology creation and maintenance directly within text editors using e.g., Turtle or JSON-LD.
This approach is especially beneficial for quick iterations and avoids the overhead of additional software installation and configuration.
Lastly, for data engineers crafting SPARQL queries, SWLS intends to help them with intelligent autocompletion, which suggests properties and classes based on context, and hover-based explanations that help users understand ontology structures.
These features are intended to reduce errors, reduce the learning curve and in general, promote a deeper engagement with Semantic Web technologies.
Concretely SWLS supports:
\begin{itemize}
  \item Turtle, JSON-LD and SPARQL
  \item Semantic and syntactic highlighting
  \item Suggest autocompletions from properties (ordered on domain), classes, already defined prefixes and suggestions for new prefixes coming from Prefix.cc
  \item Show diagnostics, including undefined prefixes, incorrect syntax and shape violations
  \item Formatting and renaming
  \item Hover functionality showing additional type information
\end{itemize}

Previous efforts, such as YASGUI for SPARQL query editing, have shown the value of lowering entry barriers by improving usability~\cite{10.3233/SW-150197,10.1007/978-3-642-41242-4_7}. 
SWLS builds on these successes, expanding support to multiple Semantic Web languages and emphasizing a user-friendly, extensible design. 
By addressing the needs of our user groups, we aim to advance the state of the art in Semantic Web tooling and foster its broader adoption.

The reaminder of this paper, presents first in Section~\ref{sec:related_work} the state of the art in tooling currently available to different user groups.
We then introduce in Section~\ref{sec:semantic_lsp} the SWLS covering its capabilities and our implementation approach.
Then in Section~\ref{sec:usage} we discuss the impact of SWLS for different user groups,
and lastly in Section~\ref{sec:conclusion} we present our conclusions and targets for future work.

